% Options for packages loaded elsewhere
% Options for packages loaded elsewhere
\PassOptionsToPackage{unicode}{hyperref}
\PassOptionsToPackage{hyphens}{url}
\PassOptionsToPackage{dvipsnames,svgnames,x11names}{xcolor}
%
\documentclass[
  letterpaper,
  DIV=11,
  numbers=noendperiod]{scrartcl}
\usepackage{xcolor}
\usepackage{amsmath,amssymb}
\setcounter{secnumdepth}{-\maxdimen} % remove section numbering
\usepackage{iftex}
\ifPDFTeX
  \usepackage[T1]{fontenc}
  \usepackage[utf8]{inputenc}
  \usepackage{textcomp} % provide euro and other symbols
\else % if luatex or xetex
  \usepackage{unicode-math} % this also loads fontspec
  \defaultfontfeatures{Scale=MatchLowercase}
  \defaultfontfeatures[\rmfamily]{Ligatures=TeX,Scale=1}
\fi
\usepackage{lmodern}
\ifPDFTeX\else
  % xetex/luatex font selection
\fi
% Use upquote if available, for straight quotes in verbatim environments
\IfFileExists{upquote.sty}{\usepackage{upquote}}{}
\IfFileExists{microtype.sty}{% use microtype if available
  \usepackage[]{microtype}
  \UseMicrotypeSet[protrusion]{basicmath} % disable protrusion for tt fonts
}{}
\makeatletter
\@ifundefined{KOMAClassName}{% if non-KOMA class
  \IfFileExists{parskip.sty}{%
    \usepackage{parskip}
  }{% else
    \setlength{\parindent}{0pt}
    \setlength{\parskip}{6pt plus 2pt minus 1pt}}
}{% if KOMA class
  \KOMAoptions{parskip=half}}
\makeatother
% Make \paragraph and \subparagraph free-standing
\makeatletter
\ifx\paragraph\undefined\else
  \let\oldparagraph\paragraph
  \renewcommand{\paragraph}{
    \@ifstar
      \xxxParagraphStar
      \xxxParagraphNoStar
  }
  \newcommand{\xxxParagraphStar}[1]{\oldparagraph*{#1}\mbox{}}
  \newcommand{\xxxParagraphNoStar}[1]{\oldparagraph{#1}\mbox{}}
\fi
\ifx\subparagraph\undefined\else
  \let\oldsubparagraph\subparagraph
  \renewcommand{\subparagraph}{
    \@ifstar
      \xxxSubParagraphStar
      \xxxSubParagraphNoStar
  }
  \newcommand{\xxxSubParagraphStar}[1]{\oldsubparagraph*{#1}\mbox{}}
  \newcommand{\xxxSubParagraphNoStar}[1]{\oldsubparagraph{#1}\mbox{}}
\fi
\makeatother


\usepackage{longtable,booktabs,array}
\usepackage{calc} % for calculating minipage widths
% Correct order of tables after \paragraph or \subparagraph
\usepackage{etoolbox}
\makeatletter
\patchcmd\longtable{\par}{\if@noskipsec\mbox{}\fi\par}{}{}
\makeatother
% Allow footnotes in longtable head/foot
\IfFileExists{footnotehyper.sty}{\usepackage{footnotehyper}}{\usepackage{footnote}}
\makesavenoteenv{longtable}
\usepackage{graphicx}
\makeatletter
\newsavebox\pandoc@box
\newcommand*\pandocbounded[1]{% scales image to fit in text height/width
  \sbox\pandoc@box{#1}%
  \Gscale@div\@tempa{\textheight}{\dimexpr\ht\pandoc@box+\dp\pandoc@box\relax}%
  \Gscale@div\@tempb{\linewidth}{\wd\pandoc@box}%
  \ifdim\@tempb\p@<\@tempa\p@\let\@tempa\@tempb\fi% select the smaller of both
  \ifdim\@tempa\p@<\p@\scalebox{\@tempa}{\usebox\pandoc@box}%
  \else\usebox{\pandoc@box}%
  \fi%
}
% Set default figure placement to htbp
\def\fps@figure{htbp}
\makeatother


% definitions for citeproc citations
\NewDocumentCommand\citeproctext{}{}
\NewDocumentCommand\citeproc{mm}{%
  \begingroup\def\citeproctext{#2}\cite{#1}\endgroup}
\makeatletter
 % allow citations to break across lines
 \let\@cite@ofmt\@firstofone
 % avoid brackets around text for \cite:
 \def\@biblabel#1{}
 \def\@cite#1#2{{#1\if@tempswa , #2\fi}}
\makeatother
\newlength{\cslhangindent}
\setlength{\cslhangindent}{1.5em}
\newlength{\csllabelwidth}
\setlength{\csllabelwidth}{3em}
\newenvironment{CSLReferences}[2] % #1 hanging-indent, #2 entry-spacing
 {\begin{list}{}{%
  \setlength{\itemindent}{0pt}
  \setlength{\leftmargin}{0pt}
  \setlength{\parsep}{0pt}
  % turn on hanging indent if param 1 is 1
  \ifodd #1
   \setlength{\leftmargin}{\cslhangindent}
   \setlength{\itemindent}{-1\cslhangindent}
  \fi
  % set entry spacing
  \setlength{\itemsep}{#2\baselineskip}}}
 {\end{list}}
\usepackage{calc}
\newcommand{\CSLBlock}[1]{\hfill\break\parbox[t]{\linewidth}{\strut\ignorespaces#1\strut}}
\newcommand{\CSLLeftMargin}[1]{\parbox[t]{\csllabelwidth}{\strut#1\strut}}
\newcommand{\CSLRightInline}[1]{\parbox[t]{\linewidth - \csllabelwidth}{\strut#1\strut}}
\newcommand{\CSLIndent}[1]{\hspace{\cslhangindent}#1}



\setlength{\emergencystretch}{3em} % prevent overfull lines

\providecommand{\tightlist}{%
  \setlength{\itemsep}{0pt}\setlength{\parskip}{0pt}}



 


\KOMAoption{captions}{tableheading}
\makeatletter
\@ifpackageloaded{caption}{}{\usepackage{caption}}
\AtBeginDocument{%
\ifdefined\contentsname
  \renewcommand*\contentsname{Table of contents}
\else
  \newcommand\contentsname{Table of contents}
\fi
\ifdefined\listfigurename
  \renewcommand*\listfigurename{List of Figures}
\else
  \newcommand\listfigurename{List of Figures}
\fi
\ifdefined\listtablename
  \renewcommand*\listtablename{List of Tables}
\else
  \newcommand\listtablename{List of Tables}
\fi
\ifdefined\figurename
  \renewcommand*\figurename{Figure}
\else
  \newcommand\figurename{Figure}
\fi
\ifdefined\tablename
  \renewcommand*\tablename{Table}
\else
  \newcommand\tablename{Table}
\fi
}
\@ifpackageloaded{float}{}{\usepackage{float}}
\floatstyle{ruled}
\@ifundefined{c@chapter}{\newfloat{codelisting}{h}{lop}}{\newfloat{codelisting}{h}{lop}[chapter]}
\floatname{codelisting}{Listing}
\newcommand*\listoflistings{\listof{codelisting}{List of Listings}}
\makeatother
\makeatletter
\makeatother
\makeatletter
\@ifpackageloaded{caption}{}{\usepackage{caption}}
\@ifpackageloaded{subcaption}{}{\usepackage{subcaption}}
\makeatother
\usepackage{bookmark}
\IfFileExists{xurl.sty}{\usepackage{xurl}}{} % add URL line breaks if available
\urlstyle{same}
\hypersetup{
  pdftitle={``Elected on Friday, Assassinated on Saturday'': Text-Analyzing Political Assassinations},
  colorlinks=true,
  linkcolor={blue},
  filecolor={Maroon},
  citecolor={Blue},
  urlcolor={Blue},
  pdfcreator={LaTeX via pandoc}}


\title{``Elected on Friday, Assassinated on Saturday'': Text-Analyzing
Political Assassinations}
\author{Jeff Jacobs}
\date{}
\begin{document}
\maketitle
\begin{abstract}
In the historiography of political thought, radical thinkers are often
characterized as those who ``push the boundaries'' of the range of
acceptable ideas and opinions in a given society at a given time. While
formal modeling of strategic ``spatial voting'' in political science has
led to influential results like the Median Voter Theorem, less work has
been devoted to assessing strategic behavior among political thinkers at
the ``margins'' of political thought, whose aim is not to
\emph{represent} public opinion (and thus win votes) but to
\emph{influence it}. In this work, therefore, we posit a ``Median
Survivor Theorem'', a point on a given ideological spectrum beyond which
influential political thinkers are vastly more likely to be
assassinated, and then estimate this point via a text-analysis of the
topical trajectory of prominent radical thinkers' speeches before and
after assassination \emph{attempts} and, especially, before successful
assassinations. After demonstrating the efficacy of this approach in
terms of ability to predict assassinations from textual content, we zoom
in on individual cases, measuring the ideological dissimilarities
between thinkers' public and private statements and assessing the degree
to which their public ``position-taking'' exhibited strategic
compromises with broader social-discursive boundaries.
\end{abstract}


\begin{quote}
\emph{If I was president, I'd get elected on Friday, assassinated on
Saturday, buried on Sunday, then go back to work on Monday.}

Wyclef Jean (2004)
\end{quote}

\subsection{Introduction}\label{introduction}

Spatial metaphors abound in the historiography of political thought.
Political scientists, for example, often employ the notion of an
``Overton Window'' to denote the range of acceptable ideas and opinions
within a given society at a given time, such that ``radical'' thinkers
can be characterized in turn as those who push the boundaries of this
range.

These metaphors then underlie a slew of formal and empirical research
endeavors in the social sciences, such as ``ideal point estimation''
(Poole and Rosenthal 1985), whereby voters' positions along the Overton
Window are represented as points on the real numberline
(\(\mathbb{R}\)), while electoral candidates strategically choose
platform positions with respect to the range of their constituents'
ideological points. Several influential results emerge from this
framework, such as the Median Voter Theorem, which states that two
``rational'' candidates competing for votes from the same constituency
will adopt nearly-identical platforms (differing by an
infinitesimally-small magnitude \(\varepsilon\)), at exactly the median
of their constituents' ideological points in \(\mathbb{R}\), since any
move away from this point will reduce the proportion of captured
candidates below 50\%.

In this work, we employ this framework to analyze the \emph{strategy} of
radical thinkers, and find that indeed there is a ``Median Radicalism''
point whereby those thinkers whose speech moves beyond this point
drastically increase their likelihood of being assassinated. After
presenting our methodology and findings, we then use more ``in-depth'',
small-\(N\) evidence to assess the degree to which radical thinkers make
these strategic choices in a \emph{conscious} manner: that is, the
extent to which broader cultural \emph{``enforcement''} of the Overton
Window (e.g., in media or activist discourse), in conjunction with the
fear of assassination, leads such thinkers to curb their radicalism in
public statements relative to their private writings (e.g., in personal
journals or correspondence with fellow radical thinkers).

\subsection{Data (MVP Version)}\label{data-mvp-version}

As an instance of developing a ``Minimum Viable Product'' (MVP), we
begin our study in an exploratory mode. We operationalize popular
notions that Martin Luther King's foray into particular strands of
discourse -- concretely, his increasingly-urgent calls to his
constituency to focus on and oppose the United States' genocides in
Vietnam and Indochina more broadly -- ``led to'' his
assassination\footnote{Our emphasis in this initial stage thus
  ``dovetails'' with a related body of historical research on the tactic
  of nonviolence in the 1960s Civil Rights struggle. See, especially,
  Cobb (2014).}. At this stage, however, we also note that ``stopping''
with just the single case of Martin Luther King would subject our study
to the bias of \emph{selection on the dependent variable}. Thus, while
we choose cases more methodically in Section~\ref{sec-methods} using a
statistical matching approach, here we choose Black Panther Party
chairman Huey Newton as our ``control'' case (Newton 1973), as a
political thinker whose radical thought also brought him to the
forefront of the black liberation movement only a few years after King's
murder, but who was not assassinated until decades later\footnote{Newton
  was murdered in 1989, at the age of 47, in Oakland, California.}.

\subsection{Methodology}\label{sec-methods}

\subsection{Results}\label{results}

\subsection{Conclusion}\label{conclusion}

\subsection{References}\label{references}

\phantomsection\label{refs}
\begin{CSLReferences}{1}{0}
\bibitem[\citeproctext]{ref-cobb_this_2014}
Cobb, Charles E. 2014. \emph{This {Nonviolent Stuff}'ll {Get You
Killed}: {How Guns Made} the {Civil Rights Movement Possible}}. Basic
Books.

\bibitem[\citeproctext]{ref-newton_revolutionary_1973}
Newton, Huey P. 1973. \emph{Revolutionary {Suicide}: ({Penguin Classics
Deluxe Edition})}. Penguin.

\bibitem[\citeproctext]{ref-poole_spatial_1985}
Poole, Keith T., and Howard Rosenthal. 1985. {``A {Spatial Model} for
{Legislative Roll Call Analysis}.''} \emph{American Journal of Political
Science} 29 (2): 357--84. \url{https://doi.org/10.2307/2111172}.

\bibitem[\citeproctext]{ref-wyclefjean_president_2004}
Wyclef Jean. 2004. {``President.''}
\url{https://www.youtube.com/watch?v=9pq_3OheqzU}.

\end{CSLReferences}

\subsection{Corpus}\label{corpus}

\phantomsection\label{corpus}




\end{document}
